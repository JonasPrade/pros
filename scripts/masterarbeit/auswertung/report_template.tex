\subsection{\VAR{scenario_name}}

\subsubsection{Kenngrößen Szenario}

\captionof{table}{\label{table_\VAR{scenario_id}_kenngrößen_scenario} Basiskenngrößen Szenario \VAR{scenario_id}}
\begin{center}
	\begin{tabularx}{\textwidth}{X | X } Kenngröße & Wert \\
	\hline
	Anzahl Untersuchungsgebiete & \num{\VAR{count_master_area}} \\
	Summe Streckenkilometer & \SI{\VAR{length_scenario|round}}{\km}
	\end{tabularx}
\end{center}

\subsubsection{Kenngrößen Untersuchungsgebiete}
\captionof{table}{\label{table_\VAR{scenario_id}_kenngrößen_untersuchungsgebiete} Anzahl Untersuchungsgebiete sowie Kilometer nach Traktion \VAR{scenario_id}}
\begin{center}
	\begin{tabularx}{\textwidth}{X | X | X} Traktion & Anzahl & Kilometer Infrastruktur & Betriebskilometer \\
	\hline
    \BLOCK{ for key in cost_effective_traction }
        \VAR{key | replace("_", " ")} & \num{\VAR{cost_effective_traction[key]['area']}} &  \SI{\VAR{(cost_effective_traction[key]['infra_km'])|round}}{\km} & \SI{\VAR{(cost_effective_traction[key]['running_km'])|round}}{\km}\\
    \BLOCK{ endfor }
	\end{tabularx}
\end{center}

\captionof{table}{\label{table_\VAR{scenario_id}_kenngrößen_untersuchungsgebiete_no_optimised} Anzahl Untersuchungsgebiete sowie Kilometer nach Traktion (Optimierte Elektrifizierung aufgeteilt) Szenario \VAR{scenario_id}}
\begin{center}
	\begin{tabularx}{\textwidth}{X | X | X} Traktion & Anzahl & Kilometer \\
	\hline
    \BLOCK{ for key in cost_effective_traction_no_optimised }
        \VAR{key | replace("_", " ")} & \num{\VAR{cost_effective_traction_no_optimised[key]['area']}} &  \SI{\VAR{(cost_effective_traction_no_optimised[key]['infra_km'])|round}}{\km} & \SI{\VAR{(cost_effective_traction_no_optimised[key]['running_km'])|round}}{\km}\\
    \BLOCK{ endfor }
	\end{tabularx}
\end{center}
In dieser Tabelle wurden die Infrastrukturkilometer der Untersuchungsgebiete, bei denen die optimierte Elektrifizierung am wirtschaftlichsten war, auf die Traktionen Elektrifizierung und Batterie aufgeteilt (je nachdem, bei welchem Teiluntersuchungsgebiet welche Traktion am wirtschaftlichsten war)

\subsubsection{Kenngrößen Kosten}

\captionof{table}{\label{table_\VAR{scenario_id}_kenngrößen_untersuchungsgebiete_no_optimised} Investitionsvolumen sowie Betriebskosten (Barwert Preisstand 2016) Szenario \VAR{scenario_id}}
\begin{center}
	\begin{tabularx}{\textwidth}{X | X | X} Traktion & Kosten Infrastruktur [Tsd. € Barwert] & Betriebskosten [Tsd. € Barwert]\\
	\hline
    \BLOCK{ for key in cost_effective_traction }
        \VAR{key | replace("_", " ")} & \num{\VAR{(cost_effective_traction[key]['infrastructure_cost'])|round}} &  \num{\VAR{(cost_effective_traction[key]['operating_cost'])|round}}\\
    \BLOCK{ endfor }
	\end{tabularx}
\end{center}

\begin{center}
	\begin{figure}[!h]
	\includegraphics[height=0.8\paperheight]{\VAR{filepath_d_map}}
	\caption{\label{fig_s_\VAR{scenario_id}_d_map} Karte Deutschland bevorzugte Traktion Untersuchungsgebiete Szenario \VAR{scenario_id}}
	\end{figure}
\end{center}
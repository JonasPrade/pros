\subsubsection{\VAR{pc.name}}\label{chap_resilience_\VAR{pc.id}}
Bevorzugtes Szenario: \VAR{preferred_scenario}

\captionof{table}{\label{table_resilience_\VAR{pc.id}} Kenngrößen \VAR{pc.name}}
\begin{center}
	\begin{tabularx}{\textwidth}{l | X } Kenngröße & Wert \\
		\hline
		Sperrung & \VAR{additional_data["endpoints_closure"][0]} – \VAR{additional_data["endpoints_closure"][1]} \\
		Anzahl Güterzüge pro Tag & \VAR{count_trains} \\
		Zugkilometer Referenzszenario pro Tag & \num{\VAR{running_km_day_reference_scenario|round|int}} \\
		Zugkilometer Resilienzszenario pro Tag & \num{\VAR{running_km_day_resilience_scenario|round|int}} \\
		Streckenlänge nicht elektrifiziert [km] & \num{\VAR{length_new_rw_lines|round|int}} \\
		Störungshäufigkeit & \VAR{parameter.DISTURBANCE_PERCENTAGE}
	\end{tabularx}
\end{center}
\hspace{2em}

\captionof{table}{\label{table_resilience_cost_\VAR{pc.id}} Kostenvergleich \VAR{pc.name}}
\begin{center}
	\begin{tabularx}{\textwidth}{X | X | X} Kenngröße & Variante Straße & Variante Umfahrung Schiene \\ \hline
		Gesamtkosten Betrachtungszeitraum [Tsd. €] & \num{\VAR{cost["cost_road_case"]|round|int}} & \num{\VAR{cost["cost_resilience"]|round|int}} \\
		Kosten Untersuchungsgebiete Betrachtungszeitraum [Tsd. €] & \num{\VAR{cost["areas_ref_scenario"]|round|int}} & \num{\VAR{cost["areas_res_scenario"]|round|int}} \\
		Veränderung Betriebskosten Betrachtungszeitraum [Tsd. €] & \num{\VAR{cost["road_coast_operation_duration"]|round|int}} & \num{\VAR{cost["operating_cost_sgv_resilience_sum"]|round|int}} \\
		Veränderung Betriebskosten Tag [Tsd. €] & \num{\VAR{cost["road_cost_day"]|round|int}} & \num{\VAR{cost['operating_cost_sgv_resilience_day']|round|int}}
	\end{tabularx}
\end{center}
\hspace{2em}

\begin{center}
	\begin{figure}[p]
	\centering
	\includegraphics[height=\textheight, width=\textwidth, keepaspectratio]{\VAR{filepath_map}}
	\caption{\label{fig_pc_\VAR{pc.id}} Karte Resilienszenario \VAR{pc.name}}
	\textit{In blau die gesperrte Strecke. Die befahrene Strecken für den Güterverkehr im Umleitungsfall in rot (davor nicht elektrifiziert) und grün (davor bereits elektrifiziert), in Dicke die Anzahl der Fahrten}
	\end{figure}
\end{center}

